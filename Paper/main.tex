\documentclass[letterpaper, 10 pt, conference]{ieeeconf}  % Comment this line out if you need a4paper

%\documentclass[a4paper, 10pt, conference]{ieeeconf}      % Use this line for a4 paper

\IEEEoverridecommandlockouts                              % This command is only needed if 
                                                          % you want to use the \thanks command

\overrideIEEEmargins                                      % Needed to meet printer requirements.

%\renewcommand{\baselinestretch}{2}
\pdfminorversion=4
\usepackage{tikz}
\usetikzlibrary{shapes,arrows}



% See the \addtolength command later in the file to balance the column lengths
% on the last page of the document
\newtheorem{theorem}{Theorem}[section]
\newtheorem{corollary}{Corollary}[theorem]
\newtheorem{lemma}[theorem]{Lemma}
\usepackage{amsmath} 
\usepackage{amssymb} 


\usepackage{ltl} 

\usepackage{algorithm2e}

\usepackage{color}
\usepackage{subfig}
\usepackage{tikz}
\usetikzlibrary{arrows,fit,shapes,automata}
\usetikzlibrary{positioning,fit,calc,shapes}
\usetikzlibrary{decorations.fractals}
\usetikzlibrary{decorations.markings}

\newcommand{\Rayna}[1]{{\textcolor{magenta}{ \textbf{Rayna:} #1 $\spadesuit$ }}}
\newcommand{\Suda}[1]{{\textcolor{blue}{ \textbf{Suda:} #1 $\spadesuit$ }}}
\newcommand{\Ufuk}[1]{{\textcolor{red}{ \textbf{Ufuk:} #1 $\spadesuit$ }}}
\newcommand{\todo}[1]{{\textcolor{red}{TODO:} #1}}

% Local macros
\providecommand{\st}{\mathrel{\mid}}
\newcommand{\defeq}{:=}
\newcommand{\pow}[1]{2^{#1}}
\newcommand{\supp}{\mathsf{supp}}
\newcommand{\reach}{\mathsf{Reach}}
\newcommand{\dist}[1]{\mathcal{D}(#1)}
\newcommand{\ie}{\textit{i.e.}\xspace}
\newcommand{\eg}{\textit{e.g.}\xspace}
\newcommand{\myparagraph}[1]{\par\smallskip\noindent\textbf{#1.}}
\newcommand{\prob}{\mathcal{P}}
\newcommand{\polreachp}[4]{\prob_{#1}^{#2,#3}[\mathrm{Reach}(#4)]}
\newcommand{\maxreachp}[3]{\mathrm{Val}_{#1}^{#2}(#3)}


\newtheorem{example}{Example}

\newcommand{\init}{\mathsf{init}}
\newcommand{\belief}{\mathsf{belief}}
\newcommand{\abstr}{\mathsf{abstract}}
\newcommand{\vis}{\mathit{vis}}
\newcommand{\succs}{\mathit{succ}}
\newcommand{\beliefs}{\mathcal{P}(L_t)}

\newcommand{\Surveillance}{\mathsf{Surveillance}}
\newcommand{\beliefF}{\mathit{belief}}

\newcommand{\states}{S}
\newcommand{\trans}{T}
\newcommand{\part}{\mathcal{Q}}

\newcommand{\post}{\mathit{post}}

\newcommand{\outcome}{\mathit{outcome}}
\newcommand{\counterex}{\mathcal{C}}

\newcommand{\bools}{\mathbb{B}}
\newcommand{\true}{\mathit{true}}
\newcommand{\false}{\mathit{false}}
\newcommand{\nats}{\mathbb{N}}

\newcommand{\SP}{\mathcal{SP}}
\newcommand{\AP}{\mathcal{AP}}
\DeclareMathOperator*{\argmin}{arg\,min}




   
\def\fat#1{#1}

\title{Transfer Entropy MDPs with  Temporal Logic Specifications
}

\author{Suda Bharadwaj \and Mohamadreza Ahmadi \and
 Takashi Tanaka \and Ufuk Topcu% <-this % stops a space
%\thanks{*This work was not supported by any organization}% <-this % stops a space
\thanks{All authors are with the University of Texas at Austin. E-mail: \{suda.b, mrahmadi, ttanaka, utopcu\}@utexas.edu}%
%\thanks{$^{2}$Bernard D. Researcheris with the Department of Electrical Engineering, Wright State University, Dayton, OH 45435, USA {\tt\small b.d.researcher@ieee.org}}%
}


\tikzstyle{block} = [draw, fill=blue!20, rectangle, 
    minimum height=3em, minimum width=6em]
\tikzstyle{sblock} = [draw, fill=blue!20, rectangle, 
    minimum height=3em, minimum width=3em]
\tikzstyle{sum} = [draw, fill=blue!20, circle, node distance=1.2cm]
\tikzstyle{input} = [coordinate]
\tikzstyle{output} = [coordinate]
\tikzstyle{pinstyle} = [pin edge={to-,thin,black}]



\begin{document}

\maketitle
\thispagestyle{empty}
\pagestyle{empty}

%%%%%%%%%%%%%%%%%%%%%%%%%%%%%%%%%%%%%%%%%%%%%%%%%%%%%%%%%%%%%%%%%%%%%%%%%%%%%%%%

\begin{abstract}
Applications in autonomy, in particular, space rovers motivate the need for control techniques that take into account uncertain environments, communication, and sensing constraints, while satisfying high-level mission specifications. In order to address such  challenges, we consider a class of Markov decision processes (MDPs), where the cost function is \emph{transfer entropy} which we refer to as TEMDPs. In this context, we study high-level mission specifications in terms of co-safe linear temporal logic (LTL) formulae.   Based on the dynamic programming principle, we provide a method to synthesize a policy that minimizes the weighted sum of the cost function and the probability of failure to satisfy the LTL specification. We show that this policy can be obtained using a modified Arimoto-Blahut algorithm. Finally, we show the efficacy of the proposed method by applying it to address navigation and path planning scenarios of a Mars rover, demonstrating the effect of considering the transfer entropy~cost.
\end{abstract}

%%%%%%%%%%%%%%%%%%%%%%%%%%%%%%%%%%%%%%%%%%%%%%%%%%%%%%%%%%%%%%%%%%%%%%%%%%%%%%%%

\section{Introduction}
Autonomous systems are expected to complete increasingly more complex missions in dynamic and uncertain environments. In many domains however, these missions are often limited by communication or sensing restrictions. Consider a Mars rover that has been tasked to safely explore and coordinate with a scouting helicopter as proposed in \cite{landau2015helicopter} for the upcoming 2020 Mars rover mission. Missions of such sophisticated nature will necessitate on-board autonomy \cite{francis2017advanced,estlin2007increased}. There are, however, tight sensing restrictions due to the power consumption of on-board sensors and transmitters. Furthermore, there are bandwidth constraints on data sent from the Earth and orbiting satellites \cite{sherwood2014,Backes1999}. In these cases it is necessary to have controllers that can make use of \emph{limited information} and still complete their mission specification above an acceptable threshold. %In some situations, there is not enough time to transmit all sensor data before a control decision can be made. For example, in distributed control of UAVs that may need to react quickly to changes in the environment \cite{Baillieul07}. 

In this paper, we model the environment and plant dynamics using a Markov decision process (MDP). MDPs are one of the most widely studied models for decision-making under uncertainty in the fields of artificial intelligence, robotics, and optimal control \cite{Papadimitriou87,Fu15}. Hence, understanding the effects of limiting the information flow to the policy synthesizer is key towards pushing the use of MDPs in information-constrained applications like Mars rovers. This will allow us to take advantage of the existing well studied techniques in controlling MDPs for use in communication and sensing constrained situations. 

Defining sophisticated mission objectives in MDPs requires specifying complex reward functions in order to synthesize a control policy \cite{puterman2014}. Alternatively, temporal logic has been used as a formal way to allow the user to more intuitively specify high-level specifications such as infinitely often patrolling a region or moving through certain regions in a specific order. Linear temporal logic (LTL) in particular has been popular for use in optimal control \cite{Svoreňová13,Fu15} as it is widely studied and several tools exist to synthesize provably correct policies in MDPs. %It is known that deriving controllers for rich temporal logic specifications reduces to solving a reachability problem on an MDP. 
The focus of related work in this field is in cases where full state information is \emph{unavailable}, \ie where only part of the state can be observed, whereas we look at scenarios where the information is available, but with restricted access. Specifically, we study the effect of information restriction on satisfying temporal logic objectives in MDPs.

In order to study the effects of restricted information flow, we need to be able to quantify it. We use \emph{transfer entropy} \cite{schreiber2000} to quantify the directional information flow between two random processes, for example, from the state of an MDP to the control policy. Intuitively minimizing the transfer entropy promotes policies that rely less on knowledge of the current state of the system. In communication theory, a related quantity called \emph{directed information} has been used to measure channel capacities in feedback systems \cite{massey1990causality,tatikonda2009capacity} as well as a proxy for feedback data rate to controllers \cite{silva2011achievable}. However, transfer entropy is often used for studying causality in fields including statistical thermodynamics \cite{parrondo2015thermodynamics}, neuroscience \cite{vicente2011transfer}, and biology \cite{tung2007inferring}. Transfer entropy has also been previously used with policy synthesis in MDPs in \cite{takashi17}, though without temporal logic specifications. 

There has been a lot of work on quantifying information requirements for low-level control requirements like stability \cite{Nair07}. However, quantifying information requirements for high level decision making scenarios that we are interested in are not as widely studied. This is analogous to model reduction techniques for MDPs studied in \cite{Bharadwaj17,brazdil2014verification,ciesinski2008reduction}, where states and actions that are completely irrelevant to the mission are removed. However, information use is not quantified so there may still be an overreliance on the existing information. ~\cite{Tishby2011} examines directed information in MDPs to quantify information. Policies are penalized if they vary too much from a completely uninformed starting point, e.g, take any action with equal probability. We are, however, interested in studying the causality of information from the state to the controller, i.e, we seek to not send information that is not needed for the decision-making process. This property makes transfer entropy a better information-theoretic metric for our setting. Our work can be seen as a generalization of \cite{takashi17}, where the transfer entropy is also used, but the authors do not allow the possibility of penalizing certain sensors over others. This is crucial in settings where different sensors can have varying energy costs and is hence necessary to budget the use of certain sensors to conserve energy. Furthermore, in distributed control of multiple agents, it is sensible to reduce the reliance of information transmitted from other agents compared to information from on-board sensors due to bandwidth restrictions in transmission.

\paragraph*{\textbf{Contributions}} We develop formal means to connect information-theoretic techniques for policy synthesis in MDPs with techniques from formal methods and probabilistic model checking. Specifically, we include a transfer entropy cost in that we  minimize along with probability of failure of satisfying a probability specification. 

In contrast to standard MDP policy computation under temporal logic specifications, the additional information cost leads to randomized optimal policies \cite{tanaka2017lqg,Todorov09,takashi17}. This necessitates policy search in an infinite state space. To solve this efficiently, we derive a sufficient optimality condition in the form of coupled non-linear equations. We solve this using a modified version of a forward-backward iterative algorithm from ~\cite{Blahut72}. While the proposed method builds on earlier results in \cite{takashi17}, we generalize the setting to penalize subsets of state variables and incorporate rich temporal logic constraints. 

Additionally, in the application of networked control systems, we provide a physical interpretation of the transfer entropy cost. We prove that it is in fact the lower bound on the data rate of the penalized state variables that must be transmitted.

Finally, we present a path planning example in grid worlds with moving and static obstacles and qualitatively analyze the impact of the information cost on relevant state variables.

% Control systems using multiple distributed sensors and/or processors has become increasingly popular in recent times \cite{Baillieul07}. In such scenarios, understanding communication restrictions and delays is imperative for real world applications. Indeed, a panel on future directions in control had identified control under communication restrictions to be an important future direction \cite{Murray03}.

% Accounting for cost of information is crucial in many decision-making scenarios. For example, a planetary rover may need to communicate with a ground station or orbiting satellite in order to complete its mission \cite{Backes1999}. However, there are often tight bandwidth constraints so it is necessary to only transmit the minimal amount of information possible to achieve the required performance \cite{sherwood2014}. Another example is coordinating autonomous agents in a hostile environment. In these cases it can be desirable to restrict communication to avoid detection, as well as to save power. Often, there is not enough time to transmit all sensor data before a control decision can be made such as in distributed control of UAVs that may need to react quickly to changes in the environment \cite{Baillieul07}. In these situations it is necessary to send the minimal information possible while still guaranteeing a certain amount of performance.

% Specifically, we want to synthesize a control policy in a Markov decision process (MDP) that minimizes the information reliance from the state but still performs above a minimum prescribed performance threshold. MDPs are one of the most widely studied tools for decision making under uncertainty in the fields of artificial intelligence, and robotics \cite{Papadimitriou87,Bharadwaj17} for optimal control and motion planning \cite{Ragi13,Burlet04}, so having a better understanding of information flow to controllers is key to pushing our understanding of communication restricted control. 

% We use \emph{transfer entropy} \cite{schreiber2000} to quantify the directional information flow between two random processes. Intuitively minimizing the transfer entropy promotes policies that rely less on the current state of the system. In communication theory, a related quantity called \emph{directed information} has been used to measure channel capacities in feedback systems \cite{massey1990causality,tatikonda2009capacity} as well as in studying feedback data rate to controllers \cite{silva2011achievable}. However, transfer entropy is often used when causality is being studied, usually in fields like statistical thermodynamics \cite{parrondo2015thermodynamics}, neuroscience \cite{vicente2011transfer}, and biology \cite{tung2007inferring}. Transfer entropy has also been previously used with policy synthesis in MDPs in \cite{takashi17}. 

% Defining elaborate control objectives in MDPs have been considered in \cite{puterman2014}. However, this requires specifying complex reward functions in order to synthesize a control policy. To this end, temporal logic has been used as a formal way to allow the user to more intuitively specify high level specifications. Linear temporal logic (LTL) in particular has been popular for use in optimal control \cite{Svoreňová13,Fu15} as it is widely studied and several tools exist to synthesize provably correct policies in MDPs. It is known that maximizing the probability of satisfying LTL objectives in an MDP reduces to a reachability objective ~\cite{BaierKatoen08}. 

% The focus of this paper is to minimize the transfer entropy cost from state variables in an MDP that are deemed expensive to measure or transmit to the controller whilst maintaining a minimum probability of satisfying a mission objective specified in LTL.

% The focus of this paper is situations where the state is composed of multiple state variables measured by different sensors. This is often the case in applications such as planetary rovers where different sensors can have varying energy costs and is hence crucial to budget the use of certain sensors to conserve energy. This problem can be seen as analogous to the centralized multi-agent problem, as we only want to penalize knowledge of states from other agents but not our own. It can also be seen as simply penalizing the use of certain sensors over others.

% \paragraph*{\textbf{Related work}}
% This work builds on \cite{Takashi17} where transfer entropy is also used to quantify the rate of information flow from state to the control action. However, the authors do not consider situations where the state is composed of multiple state variables measured by different sensors, not all of which are remote or expensive to measure. This is often the case in applications such as planetary rovers where different sensors can have varying energy costs and is hence crucial to budget the use of certain sensors to conserve energy. This problem can be also seen as analogous to the centralized multi-agent problem where we only want to penalize knowledge of states from other agents but not our own to limit communication between agents. 

% Early work in \textit{minimal attention} controllers was done in \cite{Brockett97} and \cite{Brockett03} where the authors implement control actions that do not require much 'monitoring'. To do this an \textit{attention} cost is used that penalizes varying control away from a constant input. \cite{Chatterjee2013} provides algorithms in a 2-player game formulation with $\omega$-regular objectives to minimize communication bandwidth requirements between processors. This was %done by treating the game as an imperfect information game with additional \textit{attention} costs and was
% shown to be EXPTIME-complete making this solution computationally infeasible in many cases. Furthermore, the cost used is once again applied to changing control action and not directly to the underlying flow of information to the controller. In our case, we do not penalize changing control action, only the access to information and this does not penalize non-trivial control laws based on limited information from the state.

% There is also related work done in the fields of optimal active sensor placement and control. ~\cite{Hoffmann10,krause2006near} use mutual information to determine the optimal time or position to sense. This is the inverse problem studied to that posed in this paper as the authors are trying to \textit{maximize} an information-theoretic term (mutual information) so the most amount of information can be obtained with minimal sensor action. This is also case in ~\cite{Naiss13} where the authors maximize directed information. Other related work such as \cite{Tishby2011} uses a Kullback-Leibler cost (as opposed to transfer entropy) that quantifies the difference in the probability distributions of controlled trajectories compared to an uncontrolled.

% \paragraph*{\textbf{Contributions}}
% We provide a formal way to connect information theoretic techniques to compute a policy in a finite state MDP that minimizes the information flow (quantified by transfer entropy) to the controller with techniques from formal methods and probabilistic model checking by including linear temporal logic constraints. Hence, our contribution generalizes existing work and also allows for more complex objectives to be achieved. As far as we are aware there is no current work that minimizes the information-theoretic cost under temporal logic constraints. As a side note, we remark that the generalized formulation proposed in this paper is needed in order to make use of the formalism of linear temporal logic.

% In contrast to standard MDP policy computation under LTL constraints, information constrained MDPs often require randomized optimal strategies \cite{tanaka2017lqg,Todorov09} and in our case we will require a policy search under an infinite state space. To this end, we exploit the structure present in the problem to derive a sufficient condition to satisfy in the form of a coupled set of nonlinear equations. We then propose a numeric forward-backward algorithm similar to that proposed in ~\cite{Blahut72} to iteratively solve these nonlinear equations. We also present path planning experiments on grid worlds using moving and static obstacles and qualitatively analyze the impact of the information cost on relevant state variables.

% Additionally, in the application of networked control systems, we provide a physical interpretation of the transfer entropy cost problem formulation. We prove that it is in fact the lower bound on the data rate of the penalized state variables that must be transmitted in order to achieve a minimal threshold of performance.

% % The contributions of this work includes:
% % \begin{itemize}
% % \item Minimizing information flow (transfer entropy) to the controller from a user specified set of state variables under linear temporal logic constraints.
% % \item Using an Arimoto-Blahut algorithm to numerically solve the optimality equations to obtain the control policy.
% % \item In the application of networked control systems, we prove that the transfer entropy cost problem formulation is in fact the lower bound on the data rate from the part of the state space that is penalized under information cost that must be transmitted in order to achieve a minimal threshold of performance.
% % \item We provide path planning experiments on grid worlds using moving obstacles and qualitatively analyse the impact of the information cost on relevant state variables.
% % \end{itemize} 
% \paragraph*{\textbf{Structure}}
% The rest of this paper will be structured as follows. In section II, we present our definitions and other notation used in this paper. We then present the formal problem statement in section III. In section IV, we derive the optimality conditions and present the numeric algorithm to compute the optimal policy in section V. In Section V we study the practical meaning of the transfer entropy cost to networked control problems, We provide experiments to analyze the impact of the information cost in section VI and analyze its impact on path planning experiments on grid worlds. Finally, we conclude in Section VII and provide future direction.
% % We want to penalize the cost of information from a \emph{subset} of the state space. Second, we will extend this to synthesize joint controllers for multiple agents in a centralized fashion. Finally, we investigate the structure and will solve for controllers in a decentralized fashion by breaking up the centralized optimization problem.

% %The following notation will be used in this paper. The sequence $\left(x_1,x_{2}...x_t \right)$ is denoted $x^t$ and the subsequence $\left(x_k,x_{k+1}...x_l \)$ is denoted by $x_{l}^{k}$. 

%\section{Case Study}\label{sec:casestudy}
%We demonstrate the use of information-constrained control of MDPs under temporal logic specifications in a scenario based on the upcoming Mars 2020 mission~\cite{landau2015helicopter}. 

Mars rovers operate in mostly unknown environments. Limited a priori knowledge of the terrain and possible obstacles can be provided from satellite imagery. This information, however, is often not enough for decision-making as was evidenced by the Curiosity rover which suffered punctures due to the unexpected presence of jagged, immobile, rocks embedded in the terrain. Proposed methods to reduce damage such as driving the rover backwards requires heavy human intervention and planning. For autonomous decision-making to be viable, information on local terrain characteristics is needed for planning so a human does not need to manually avoid obstacles. For the Mars 2020 mission, a helicopter has been proposed to act as a scout \cite{landau2015helicopter} to assist with planning. Figure \ref{fig:mars2020} shows an artists' rendering of the helicopter flying ahead to scout. The helicopter can then transmit information of the terrain back to the rover which is used for path planning.

\begin{figure}
\centering
\includegraphics[scale=0.18]{mars2020.jpg}
\caption{Artist's rendering of the proposed helicopter to scout for the Mars rover. The helicopter can fly ahead and send information back to the rover about the presence of any obstacles. \cite{landau2015helicopter}.}\label{fig:mars2020}
\end{figure}

We model a scenario where the rover is tasked with collecting samples from a specific region. The environment is modeled as an MDP as motion can be stochastic, i.e, slippage can occur. The mission is specified in LTL as it allows us to easily and formally state high level tasks for the rover, and the decision-making goal is to \emph{maximize} the probability of satisfying the mission specification. The helicopter can send terrain information (such as locations of any small rocks) to the rover to assist in path planning but transmitting information costs power. We provide a method that allows the planner to maximize the probability of completing the mission while also penalizing information transfer. This means the rover must plan to satisfy the mission specification while relying on as little information from the helicopter as possible. The specifics of the experiment is described in section $\ref{sec:exp}$.

%%%%%%%%%%%%%%%%%%%%%%%%%%%%%%%%%%%%%%%%%%%%%%%%%%%%%%%%%%%%%%%%%%%%%%%%%%%%%%%%

\section{Preliminaries}
The sequence $(x_0,x_{1}...x_t)$ is denoted $x^t$ and the subsequence $x_l,x_{k+1}...x_k$ is denoted by $x_{l}^{k}$. We use upper-case letters to denote random variables and lower-case letters for the realizations of the corresponding random variable. \\
We denote by $\dist{\mathcal{X}}$ the set of all probability distributions on a finite
set $\mathcal{X}$, \ie all functions $f: \mathcal{X} \to [0,1]$ such that $\sum_{x\in \mathcal{X}}f(x)=1$. Finally, for a set $\mathcal{S}$, we define $2^\mathcal{S}$ as the set of all subsets of $\mathcal{S}$ and $S^{\omega}$ as the set of all infinite sequences of elements in $\mathcal{S}$
\subsection{Markov decision processes}
% \paragraph{Markov decision processes}
% 	An \emph{MDP} is a tuple $M=(\mathcal{X},\mathcal{U},p)$ where $\mathcal{X}$ is a
% 	finite set of \emph{states},
% 	$\mathcal{U}$ is a finite alphabet of \emph{actions},
% 	$p: S\times\mathcal{U} \to \mathcal{D}({\mathcal{X}})$ is a (partial) \emph{probabilistic
% 	transition function} that assigns to a state $x\in \mathcal{X}$ and an action $u \in
% 	\mathcal{U}$ a probability distribution over the successor states. We
% 	abbreviate $p(x_{t},u)(x_{t+1})$ by $p(x_{t+1}|x_t,u_t)$.
\paragraph*{Labeled Markov decision process (MDP)} Consider a set $\AP$ of \emph{atomic propositions} which can be used, for example, to mark a state as being a ``faulty configuration'' (reaching it is, thus, undesirable), for example an obstacle. A \emph{labeled MDP} is an MDP whose states are labeled with atomic propositions. More formally, it is a tuple $M=(\mathcal{X},\mathcal{U},p,\AP,L)$ where 
\begin{itemize}
\item $\mathcal{X}$ is a
	finite set of \emph{states},
\item $\mathcal{U}$ is a finite alphabet of \emph{actions},
\item 	$p: \mathcal{X}\times\mathcal{U} \to \mathcal{D}({\mathcal{X}})$ is a \emph{probabilistic	transition function} that assigns, to a state $x\in \mathcal{X}$ and an action $u \in\mathcal{U}$, a probability distribution over the successor states. We abbreviate $p(x_{t},u)(x_{t+1})$ by $p(x_{t+1}|x_t,u_t)$.
\item $L : \mathcal{X} \rightarrow 2^{\AP}$ is the \emph{labeling function} which indicates the set of atomic propositions which are true in each state of the MDP.
\end{itemize}

\paragraph*{Runs and policies}
A \emph{run} from state $x_0$ with time horizon $T$ is a sequence $\rho = x_0 u_0 x_1 u_1 \dots ,x_{T-1},u_{T-1},x_{T}$ of states and actions such that for all $0 \leq t\leq T$ we have $p(x_{t+1}|x_t,u_t)>0$. 
%
A \emph{policy} corresponds to a way of selecting actions based on the history
of states and actions. While \emph{deterministic stationary} policies
are known to be sufficient for certain classes of problems, such as pure reachability ~\cite{puterman2014}, policies in general can be non-deterministic and history dependent. In this paper, we consider the general form and formally represent a policy as a conditional probability distribution $q_t(u_t|x^t,u^{t-1})$. 

A run $\rho$ is \emph{consistent} with a policy $q$ if it can be
obtained by extending its prefixes using $q$. Formally, $\rho=x_0
u_0 x_1 u_1 \dots$ is consistent with $q$ if for all $t \ge 0$ we have that
$u_t \in \{u| q_t(u|x^t,u^{t-1} > 0)\}$ and $p(x_{t+1}|x_t,u_t)>0$

\paragraph*{Markov chain}
A Markov chain is a tuple $(\mathcal{X},x_I,p)$ where $\mathcal{X}$ is (in our case) a finite set of states, $x_I \in \mathcal{X}$ is the initial state, and $p: \mathcal{X} \to \dist{\mathcal{X}}$ is a probabilistic transition function. An MDP $M$ together with a policy $q$ induces a \emph{Markov chain} $M^q$.  Notions of runs in a Markov chain are the same as those defined earlier. 

Given a Markov chain $M^q = (\mathcal{X},x_I,p)$, the state visited at the step $t$ is
a random variable. We denote by $h^{k}(x,\mathcal{B})$ the probability that a
run starting from state $x$ visits the set $\mathcal{B}$ in exactly $k$ steps. By definition
$h^{\leq i}(x,\mathcal{B}) = \sum_{k=0}^{i} h^{k}(x,\mathcal{B})$ denotes the probability that run from $x$ reaches the set $\mathcal{B}$ in \emph{at most} $i$ steps where $h^0(x,\mathcal{B})$ is $0$ if $x
\not\in \mathcal{B}$ and $1$ otherwise.

 %Furthermore, in the infinite horizon setting,
%$h(x,\mathcal{B}) = \sum_{k=0}^{\infty}h^{k}(x,\mathcal{B})$.



%\paragraph*{End components}
%An \textit{end component} of an MDP $M=(\mathcal{X},\mathcal{U},p)$
%is a pair $(\mathcal{B},\alpha)$ where $\mathcal{B} \subseteq \mathcal{X}$ and 
%$\alpha : \mathcal{B} \to 2^{\mathcal{U}}$ is a mapping from states to actions such that, by
%playing an action $\alpha(x)$ from state $x \in \mathcal{B}$, with probability $1$ the
%next state reached will also be in $T$. More formally, we require that
%for all $x \in \mathcal{B}$ it holds that
%\begin{itemize}
%	\item $\alpha(x) \in \mathcal{U}$ is non-empty;
%	\item if there are $x_t \in \mathcal{X}$ and $u \in \alpha(x)$ such that
%		$p(x_{t+1} \vert x_t,u_t ) >0$ then $x_{t+1} \in \mathcal{B}$;
%	\item for all $x,x' \in \mathcal{B}$ there is a run from $x$ going to $x'$ and a run going from $x'$ to $x$.
%\end{itemize}
%End components in an MDP can be found using graph analysis techniques ~\cite{BaierKatoen08}. 

\subsection{Temporal Logic}

\paragraph*{Specifications} We utilize linear temporal logic to specify the constraints on the system. Such specifications include invariance, safety or liveness. For example, we can specify that an agent infinitely often patrols a certain set of states (liveness) while not entering undesirable states (safety). It is known that synthesizing controllers for a rich array of temporal logic specifications reduces to a reachability problem on a lifted MDP \cite{BaierKatoen08}. In particular, we will represent specifications as a deterministic \emph{Rabin automaton} (DRA). See \cite{BaierKatoen08,safra1988complexity} for the connections between linear temporal logic specifications and their automata-based representations. 

A DRA is a tuple $\mathcal{A} = (\mathcal{S},s_I,2^{\AP}, T,\textrm{Acc})$ where $\mathcal{S}$ is a finite set of states, $\AP$ is a set of atomic propositions, $2^{\AP}$ is the alphabet of the automaton. $T: \mathcal{S} \times 2^{\AP} \rightarrow \mathcal{S} $ is the transition function and $s_I \in \mathcal{S}$ is the initial state. The acceptance condition $\textrm{Acc}$ is a set of tuples $\{(J_i,K_i) \st i= 0,1,\dots,m \}$ where $J_i,K_i \in \mathcal{S}$. Let a $w = w_0 w_1 \dots \in 2^{\AP}$ be an infinite word in the language of the automaton. A corresponding infinite run is an infinite sequence of states $s_0 w_0 s_1 w_1 \dots \in S$ where $s_0 = s_I$ and $s_{i+1} = T(s_i,w_i)$. Let $\textrm{Inf}(\rho)$ be the set of states appearing infinitely often in a run $\rho$, \ie~the set $\{s \in \mathcal{S} \st \forall i \ge 0, \exists j \ge i, s_j = s\}$. We say $\rho$ is \emph{accepting} if there exists a pair $(J_i,K_i) \in \textrm{Acc}$ such that $\textrm{Inf}(\rho) \cap J_i = \emptyset$ and $\textrm{Inf}(\rho) \cap K_i \neq \emptyset$.

\paragraph*{Product MDP}
Given an MDP $M=(\mathcal{X},\mathcal{U},p,\AP,L)$ and a specification DRA
$\mathcal{A} = (\mathcal{S},s_I,2^{\AP}, T,\textrm{Acc})$, we now define a \emph{product
MDP}, $\mathcal{M} \defeq M \mathcal{\times A}$, as $\mathcal{M}
:= (\mathcal{V},\mathcal{U}, \Delta,v_0,\textrm{ACC}_{\mathcal{M}})$ where
\begin{itemize}
	\item $\mathcal{V} = \mathcal{X} \times \mathcal{S}$;
	\item $\Delta: \mathcal{V} \times \mathcal{U} \rightarrow \dist{\mathcal{V}}$ is the probabilistic
		(partial) function with
		that $\Delta\left((x_{t+1},s_{t+1})\vert (x_t,s_t)\right) = p(x_{t+1} \vert x_t,u_t ) $ if $T(s_t,L(x_{t+1}))
		= s_{t+1}$;
	\item $v_0 = (x_0,s_I)$; is the initial state; and
	\item $\textrm{Acc}_{\mathcal{M}} =
		((\hat{J}_i,\hat{K}_i) \st i = 1,\dots,m \land \hat{J}_i = \mathcal{X}
		\times J_i,\hat{K}_i = \mathcal{X} \times K_i)$ is the accepting condition.
\end{itemize}

An end component of $\mathcal{M}$ is said to be an \textit{accepting end
component} if $W \cap \hat{J}_i = \emptyset$ and $W \cap \hat{K}_i \neq
\emptyset$ for some $(\hat{J}_i,\hat{K}_i) \in \textrm{Acc}_\mathcal{M}$.
We denote the set of accepting end components in a product MDP $\mathcal{M}$
by $\textrm{AEC}(\mathcal{M})$, and we denote the set of accepting end \textit{states} as
$\mathcal{C} := \{v \in W \st (W,x) \in \textrm{AEC}(\mathcal{M})\}$. We know that once we
enter $v \in \mathcal{W}$ and enact the corresponding policy $q$, the strategy will
ensure that, for some $(\hat{J}_i,\hat{K}_i) \in \textrm{Acc}_\mathcal{M}$, we visit $v
\in \hat{J}_i$ finitely often and $v \in \hat{K}_i$ infinitely often. Hence, the
problem of finding a policy $q$ that maximizes the probability of satisfying a
given temporal logic specification becomes a matter of synthesizing a strategy to reach a
state in $\mathcal{C}$ and once inside the set, the corresponding policy $q$ can be
followed to ensure the specification will be satisfied. Given the structure of $\mathcal{M}$, the accepting end components can be computed by algorithms in \cite{BaierKatoen08}. 

% \paragraph{T-step state value}
% Let $\mathcal{M}$ be a product MDP, $C = \textrm{AEC}(\mathcal{M})$ be the set of accepting end components and $q$ be the agent's policy in $\mathcal{M}$. We define the value of a state as the probability of reaching the accepting end component in $T$ steps or less which is equivalent to satisfying the LTL specification. Formally, for each state $v\in V$, given a finite time horizon $T \in \mathbb{N}$, the $T$-step state value at $v$ for the agent is $W^q = h^{\leq T}(v,\mathcal{C})$. 

\subsection{Information cost}
\input{infotheory}

%%%%%%%%%%%%%%%%%%%%%%%%%%%%%%%%%%%%%%%%%%%%%%%%%%%%%%%%%%%%%%%%%%%%%%%%%%%%%%%%

\section{Problem statement}
%formally introduce the problem of solving TEMDPs under co-safe LTL constraints.
%After we describe a general framework in Section~\ref{secformulation}, we discuss how the framework can incorporate the problem of synthesizing a minimal-information policy under temporal logic constraints.
In this section, we present the class of MDPs we consider and the problem statement.  

%\subsection{Mathematical formulation}
%\label{secformulation}
  A TEMDP is an MDP with a split state space $\mathcal{X} = \bar{\mathcal{X}} \times \tilde{\mathcal{X}} $ and an associated transfer entropy cost as in equation \eqref{eqdi} from $\{\bar{X}_t\}$ to $\{U_t\}$ given $\{\tilde{X}_t\}$. Formally, TEMDP is a tuple $M=(\bar{\mathcal{X}},\tilde{\mathcal{X}},\mathcal{U},p,\AP,L)$, where  $\bar{\mathcal{X}}$ denotes the 'expensive to observe' state variables, whereas  $\tilde{\mathcal{X}}$ denotes the `free' state variables.
  
Given a finite time horizon $T$ and co-safe LTL specification $\varphi$, at every time step, observing a realization of the component $\tilde{X}_t$ of the state random variable $X_t=(\bar{X}_t, \tilde{X}_t)$ is free of charge, whereas observing the `expensive' component $\bar{X}_t$ incurs cost. 

%
%Let $M=(\mathcal{X}, \mathcal{U}, p)$ be a finite MDP with state-action cost $c_t(x_t, u_t, x_{t+1})$. Let $q_t(u_t|x^t,u^{t-1})$ be the policy to be synthesized. The expected accumulated cost over a time horizon $0\leq t \leq T-1$ is denoted by
%\[
%J(X^T, U^{T-1})=\sum_{t=0}^{T-1} I(\bar{X}^t; U_t|U^{t-1},\tilde{X}^t).
%\]
As an example, imagine the MDP models a Mars rover. The value of $\bar{x}$ is measured by an orbiter or a scouting helicopter while $\tilde{x}$ is sensed using on-board sensors. Since the orbiter will only be able to communicate with the rover during part of its orbit, we will want to penalize the reliance on this information in the policy synthesis process. In the case of the helicopter we want to limit the transmission between them as this can consume power. We can represent this as a feedback control architecture shown in Figure~\ref{fig:NCS}. 
\begin{figure}
\centering
\includegraphics[width=\columnwidth]{comm.pdf}
\caption{Example of a feedback control structure with remote and local sensors}
\label{fig:NCS}
\end{figure}

At every time step $t$, assume that the free component $\tilde{X}_t$ of the state vector is immediately available to the autonomous agent, e.g, from onboard sensors of a mars rover, while the expensive component $\bar{X}_t$ is only available from a remote sensor, e.g, the scouting helicopter. We are thus interested in finding a policy $q_t(u_t|x^t,u^{t-1})$ that minimizes the information transfer from $\bar{X}$ to $U$ while ensuring specification $\varphi$ is satisfied to a given threshold $0 \leq D \leq 1$. Let $J(X^{T},U^{T-1})$ be the probability that the specification is satisfied under policy $q_t(u_t|x^t,u^{t-1})$.

%Before the state information can be transmitted, the remote sensor has to encode it into bits. The remote sensor produces a discrete codeword $W_t\in\mathcal{W}_t$ representing the state information based on a policy $p_t(w_t|\bar{x}^t, w^{t-1})$, where $\mathcal{W}_t$ is a certain codebook such that $|\mathcal{W}_t|=2^{R_t}$. The codeword is transmitted over a noiseless communication channel and received by the controller without delay, and a control input $U_t$ is generated based on a policy $p_t(u_t|\tilde{x}^t, w^t, u^{t-1})$. We aim to limiting the data transfer from the remote sensor to on-board controller. 


The main problem we study in this paper is
\begin{align}
\min_{\{q_t(u_t|x^t,u^{t-1})\}_{t=0}^{T-1}} & I(\bar{X}^{T-1}\rightarrow U^{T-1}\| \tilde{X}^{T-1}) \nonumber \\
& \textrm{s.t } J(X^{T},U^{T-1}) \geq D. \label{eqmainproblem}
\end{align}
where $\beta>0$ is a given constant.


%Assuming that transmitting data over the communication channel incurs unit cost $\beta$ per \emph{bit}, consider the following communication design problem
%\begin{equation}
%\min_{\{\mathcal{W}_t, p_t(w_t|\bar{x}^t, w^{t-1}), p_t(u_t|\tilde{x}^t, w^t, u^{t-1}) \}_{t=0}^{T-1}}  J(X^T, U^{T-1}) +\beta R \label{eqcommproblem}
%\end{equation}
%where $R=\sum_{t=0}^{T-1}R_t$ is the total amount of data. 
%The next result shows that \eqref{eqdi} provides a lower bound to $R$.
%\begin{theorem}
%\label{theorate}
%Suppose that the plant dynamics is given by an MDP $M$. For any choice of the codebook $\mathcal{W}_t$, sensor policy $p_t(w_t|\bar{x}^t, w^{t-1})$ and controller policy $p_t(u_t|\tilde{x}^t, w^t, u^{t-1})$ for $0\leq t\leq T-1$ in Figure~\ref{fig:NCS}, we have
%\[
%R \geq I(\bar{X}^{T-1}\rightarrow U^{T-1}\| \tilde{X}^{T-1}).
%\]
%\end{theorem}
%\begin{proof}
%See Appendix \ref{sec:prf}
%\end{proof}
%Theorem~\ref{theorate} shows that the optimization problem \eqref{eqmainproblem} provides a fundamental performance limitation of the networked control system shown in Figure~\ref{fig:NCS} in that the optimal value of \eqref{eqcommproblem} is lower bounded by the optimal value of \eqref{eqmainproblem}.
%
%
%A communication-theoretic meaning of the optimization problem \eqref{eqmainproblem} is discussed in section \ref{sec:ab} where we provide a physical interpretation of the transfer entropy cost.

% can be given by considering a feedback control architecture shown in Figure \ref{fig:NCS}. 
%At every time step $t$, assume that the free component $\tilde{X}_t$ of the state vector is immediately available to the controller, while the expensive component $\bar{X}_t$ is only available at a remote sensor. The remote sensor produces a discrete codeword $W_t\in\mathcal{W}_t$ based on a policy $p_t(w_t|\bar{x}^t, w^{t-1})$, where $\mathcal{W}_t$ is a certain codebook such that $|\mathcal{W}_t|=2^{R_t}$. Codeword is transmitted over a noiseless communication channel and received by the controller without delay, and a control input $U_t$ is generated based on a policy $p_t(u_t|\tilde{x}^t, w^t, u^{t-1})$. The transfer entropy cost in \eqref{eqdi} can be used to model the \emph{communication rate}. In section \ref{sec:ab}, we present a formal proof showing that \eqref{eqdi} provides a lower bound to the communication rate to justify its use in our problem formulation.

%Consider the control system architecture shown in Figure \ref{fig:NCS} modeled by a finite state discrete-time MDP. The state space $\mathcal{X}$ is composed of the product of two sets  $\mathcal{X} = \mathcal{X}_e \times \mathcal{X}_f$. Hence, each state in the state space $x \in \mathcal{X}$ is partitioned into two state variables $x = (x_e,x_f)$. Recall that a policy is a conditional policy distribution given by $q(u_t|x^t,u^{t-1})$. We can write this now as $q(u_t|x_e^t,x_f^t,u^{t-1})$. We allow the policy synthesizer full access to the value of $x_f$, but we restrict access to the value of $x_e$. 



%Let $\tilde{X} \in \mathcal{\tilde{X}}$, $\overline{X} \in \mathcal{\overline{X}}$, $U\in \mathcal{U}$ be random variables of which $\tilde{x},\overline{x},u$ are realizations. The \emph{conditional mutual information} is 
%
%\begin{align*}
%I  (X_{e_{t-m}}^t;U_t & \vert U^{t-1}_{t-n},X_{f_{t-m}}^{t}) \defeq \nonumber \\ &\sum_{\mathcal{X}^{t}_f}\sum_{\mathcal{U}^{t-1}}\log\frac{\mu_{t+1}(u_t|x_{t-m}^{t},u_{t-n}^{t-1})}{\mu_{t+1}(u_t\vert u_{t-n}^{t-1})}
%\end{align*}\todo{Fix definition}



%The transfer entropy of degree $(m,n)$ is defined as \cite{schreiber2000}
%
%\begin{align}\label{eqn:TEcond}
%I_{m,n}(X_e^T  \rightarrow & U^{T-1}||X_f^T) \defeq \nonumber \\ & \sum_{t=0}^{T-1} I\left(X_{e_{t-m}}^t;U_t|U^{t-1}_{t-n},X_{f_{t-m}}^{t} \right) .
%\end{align}
%




% An \emph{encoder} and \emph{decoder} are defined as stochastic kernels $e_t(w_t|x^t,w^{t-1})$ and $d_t(u_t|w^t,u^{t-1})$ respectively. $w_t$ is a \emph{codeword} chosen from \emph{codebook} $\mathcal{W}_t$ at time $t$. $|\mathcal{W}_t| = 2^{R_t}$. $R = \sum_{t=1}^T$ is the rate of communication. 

% For example in Figure \ref{fig:NCS}, at each time step the sensor sends data to the encoder which chooses a codeword $w_t$ and passes the message to the decoder. 


% \paragraph{Optimal policy}
% Given an MDP $M=(\mathcal{X},\mathcal{U},p)$ with cost function $c_t: \mathcal{X} \times \mathcal{U} \rightarrow \mathbb{R}$, and finite time horizon $T$, the optimal policy $q$ is one that minimizes the following

% \begin{equation}\label{eqn:optpol}
% J(X^T,U^{T-1}) \defeq \sum_{t=0}^{T-1}{\mathbb{E}\{c_t(X_t,U_t)\}} + \mathbb{E}\{c_{T}(X_{T})\}
% \end{equation}

% Informally, the policy minimizes the expected value of the cost over the time horizon. In this setting the optimal policy will be deterministic and Markovian. 

% \paragraph{Expensive-to-measure state variable} We divide the state space $\mathcal{X}$ into expensive and free to measure state variables, \ie $\mathcal{X} = \mathcal{X}_e \times \mathcal{X}_f$. A state $x \in \mathcal{X}$ can be expressed as $x = (x_e,x_f)$ where $x_e \in \mathcal{X}_e$ is expensive-to-measure and $x_f \in \mathcal{X}_f$ is free. 

% Now, consider the following information-constrained optimal control problem:
% \begin{equation}
% \min_{\{q_t\}_{t=1}^T} J(X^{T},U^{T-1}) + \beta \sum_{t=0}^T R_t
% \end{equation}
% where $\beta \in \mathbb{R}$, $R = \sum_{t=0}^T R_t$ is the rate of communication, and $J(X^T,U^{T-1}) \defeq \sum_{t=0}^{T-1}{\mathbb{E}\{c(X_t,U_t)\}} + \mathbb{E}\{c_{T}(X_{T})\}$ for some state-action dependent cost $c(X_t,U_t) \in \mathbb{R}$. 

%\begin{figure}
%\centering
%\begin{tikzpicture}[auto, node distance=2cm,>=latex']
%    % We start by placing the blocks
%    \node [input, name=input] {};
%    \node [sum, right of=input] (sum) {};
%    \node [block, right of=sum,text width=2.1cm] (controller) {\textbf{Controller}\\$q_t(u_t|x^t,u^{t-1})$};
%    \node [block, right of=controller,node distance = 3cm,text width=2cm] (dynamics) {\textbf{Dynamics}\\$p(x_{t+1}|x_t,u_t)$};
%    % We draw an edge between the controller and system block to 
%    % calculate the coordinate u. We need it to place the measurement block. 
%    \node [output, right of=dynamics] (output) {};
%    \node [sblock, below of=dynamics,text width = 1cm] (xc) {\textbf{Sensor}\\ \textit{Costly}};
%    \node [sblock, left of=xc] (encoder) {\textbf{Encoder}};
%	\node [sblock, below of=xc,node distance = 1.5cm, text width = 1cm] (xf) {\textbf{Sensor}\\\textit{Free}};
%	\node [sblock, left of=encoder,node distance = 2cm] (decoder) {\textbf{Decoder}};
%    % Once the nodes are placed, connecting them is easy. 
%    \draw [draw,->] (input) -- node {} (sum);
%    \draw [->] (sum) -- node {$X_t$} (controller);
%    \draw [->] (controller) -- node {$u_t$} (dynamics);
%    \draw [->] (dynamics) -- node [name=y] {$X_{t+1}$}(output);
%    \draw [->] (y) |- (xc);
%    \draw [->] (xc) -- node {$\bar{X}_t$} (encoder);
%    \draw [->] (y) |-  (xf);
%    \draw [->] (encoder) -- node {$W$} (decoder);
%     \draw [->] (decoder) -| node {$(\bar{X}_t,\tilde{X}_t)$} (sum);
%    \draw [->] (xf) -| node {$\tilde{X}_t$}(sum);
%\end{tikzpicture}\caption{Control system architecture where $\bar{X}_t$ is sensed remotely and has to be transmitted to the controller through an encoder-decoder system.}\label{fig:NCS}
%
%\end{figure}



\section{Incorporating temporal logic constraints}
In this section, we present a method to solve an equivalent problem to \eqref{eqmainproblem} that explicitly takes into the account the mission specification.

Consider a finite labeled TEMDP $M=(\mathcal{\hat{X}},\mathcal{U},p,\AP,L)$ where, as before, the state space of $M$ is split into expensive and cheap to measure state variables $\mathcal{\hat{X}} = \mathcal{\bar{X}}_e \times \mathcal{\tilde{X}}_f$. We are additionally given a specification DFA $\mathcal{A}_{\varphi} = (\mathcal{S},s_I,2^{\AP}, \delta,\textrm{Acc})$, and finite time horizon $T$. The product MDP is $\mathcal{M}\defeq (\mathcal{V},\mathcal{U}, \Delta,v_0,L_{\varphi},\textrm{Acc}_{\mathcal{M}})$.  Hence, we will have the state space $\mathcal{V} = (\mathcal{\bar{X}}_e \times \mathcal{\tilde{X}}_f) \times S$. Now, for notational simplicity, we set $\mathcal{X} = \mathcal{V}$, the free to measure state $\mathcal{\tilde{X}} = (\mathcal{\tilde{X}}_f,\mathcal{S})$ (we assume without loss of generality that the state in the automaton is freely known), and the expensive to measure state $\mathcal{\bar{X}} = \mathcal{\bar{X}}_e$. Let $X = (\bar{X}_e,\tilde{X}_f,S)$ and $x = (\bar{x}_f,\tilde{x}_s,s)$ be defined similarly. Thus, our state space is now $\mathcal{X}= \mathcal{\bar{X}} \times \mathcal{\tilde{X}}$ with random variable $X = (\bar{X},\tilde{X})$. 

We define a state-action cost in the product MDP in the following way. We define a function $c_t(x_t,u_t,x_{t+1})$, such that for every transition from $x_t$ to $x_{t+1}$, the cost is $0$ if neither $x_t$ or $x_{t+1}$ are in $\textrm{Acc}_{\mathcal{M}}$. The cost is $-1$ if $x_t \notin \textrm{Acc}_{\mathcal{M}}$ and $x_{t+1} \in \textrm{Acc}_{\mathcal{M}}$ and no state in $\textrm{Acc}_{\mathcal{M}}$ has been visited prior to reaching $x_t$.  Intuitively, minimizing this quantity will result in a policy $q$ that maximizes the probability of reaching $\textrm{Acc}_{\mathcal{M}}$ and hence, equivalently will maximize the probability of satisfying the temporal logic specification in $M$. The expected accumulated reward from state $x_0$ given by $\sum_{t=0}^{T-1}\mathbb{E}\{c_t(x_t,u_t,x_{t+1})\}$ will equal the \emph{negative} of the reachability probability to the target set $C$ in $T-$steps \ie we have
\vspace{-0.2cm}
\begin{align}\label{eqn:cost}
\sum_{t=0}^{T-1}\mathbb{E}\{c_t(x_t,u_t,x_{t+1})\} = -h^{\leq T}(x,\textrm{Acc}_{\mathcal{M}})
\end{align}.
\vspace{-0.3cm}
% First we define a reward function $R(v_t,u_t,v_{t+1})$, such that for every transition from $v_t$ to $v_{t+1}$, the reward is $0$ if neither $v_t$ or $v_{t+1}$ are in $C$. The reward is $1$ if $v_t \notin C$ and $v_{t+1} \in C$ and no state in $C$ have been visited prior to reaching $v_t$. Put simply, maximizing this quantity will result in a policy $q$ that maximizes the probability reaching $C$ and hence, equivalently will maximize the probability of satisfying the LTL specification $\varphi$ in $M$. The expected accumulated reward from state $v_0$ given by $\sum_{t=0}^{T-1}\mathbb{E}\{R(v_t,u_t,v_{t+1})\}$ will equal the T-step state value defined earlier. However, in this paper we are interested in solving a cost minimization problem. To do this, we define a cost function $c(v_t,u_t,v_{t+1}) = -R(v_t,u_t,v_{t+1})$. Hence, our \emph{T-step state value} under a policy $q$ will be

% \begin{align*}
% W^{q}_{\mathcal{M}} \defeq  \sum_{t=0}^{T-1}\mathbb{E}\{c(v_t,u_t,v_{t+1})\}
% \end{align*}

% which is actually the \emph{negative} of the reachability probability to the target set $C$. 

% \paragraph*{Optimal T-step policy} The optimal T-step value of the product MDP defined previously is given by $W_{\mathcal{M}}^{*}(v,T) =  \min_{q}W^q_{\mathcal{M}}(v,T)$ and the optimal T-step policy is $q^* = \argmin_{q}W^q_{\mathcal{M}}(v,T)$

% Assume now we have divided our MDP state space into an expensive-to-measure and free-to-measure state variables $\mathcal{X} = \mathcal{X}_e \times \mathcal{X}_f$. We want to find a policy $q$ in the product MDP $\mathcal{M}$ that uses minimizes the directed information from $\mathcal{X}_e$ to the policy $q$ whilst ensuring the LTL specification will be satisfied to a minimal probability threshold.

% The standard problem is to compute an optimal policy $q_t(u_t|x^t,u^{t-1})$ that minimizes the cost function as shown in equation (\ref{eqn:optpol}). We also want to minimize the rate of communication which is shown in equation (6). 

Setting $J(X^T,U^{T-1}) = \sum_{t=0}^{T-1}\mathbb{E}\{c_t(x_t,u_t,x_{t+1})\}$, we recover the formulation of \eqref{eqmainproblem}.
%We will have cost functional $J(X^T,U^{T-1}) \defeq - \sum_{t=0}^{T-1}{\mathbb{E}\{c(x_t,u_t)\}}$. Note the negative sign means that $J(X^T,U^{T-1})$ is the expected reachability probability to the target set from the initial state $x_0$. Hence we solve the optimization problem in \eqref{eqmainproblem}, on the state space of the product MDP with the cost being the negative reachability to the accepting end component. 

\paragraph*{Remark} The constrained optimization problem in equation (\ref{eqmainproblem}) can be written as a \emph{Lagrangian relaxation} in the following way

\begin{align}\label{eqn:constopt}
T_{m,n}(D) & \defeq \min_{\{q_t\}_{t=1}^T}  J(X^{T},U^{T-1}) + \beta I(\bar{X}^T \rightarrow U^{T-1}||\bar{X}^T)
\end{align}
where $\beta$ is a positive constant

Intuitively, this means that we want to minimize the information flow from the state variables in $\mathcal{\bar{X}}$ subject to the constraint on the accumulated cost $J$. Using the cost function defined in \eqref{eqn:cost}, this constrains the probability of not satisfying the specification.

% We will prove in the following that the transfer entropy cost used is a fundamental lower bound for the minimal communication rate $R(D)$, in other words, we prove that $R(D) \geq T_{m,n}(D)$. The consequence of this is that the obtained optimal control policy $\{q_t\}_{t=1}^T$ obtained as the solution to equation ($\ref{eqn:constopt}$) is one that requires a minimal data rate to implement by the controller while still achieving a control performance of at least $D$. 

%%%%%%%%%%%%%%%%%%%%%%%%%%%%%%%%%%%%%%%%%%%%%%%%%%%%%%%%%%%%%%%%%%%%%%%%%%%%%%%%

%%%%%%%%%%%%%%%%%%%%%%%%%%%%%%%%%%%%%%%%%%%%%%%%%%%%%%%%%%%%%%%%%%%%%%%%%%%%%%%%

\section{Optimality conditions}
Now, given the optimization problem in equation (\ref{eqmainproblem}), we derive sufficient conditions for optimality for the policy $q_t(u_t|x^t,u^{t-1})$. First, we look at a problem with a single time step. We use that to set up a backwards dynamic programming problem where the single-stage problem is solved at each step.
\vspace{-0.25cm}
\subsection{Single-stage problem}

\label{secsinglestage}
Let $X, U, Z$ be random variables taking values from sets $\mathcal{X}, \mathcal{U}, \mathcal{Z}$.
Let $c:\mathcal{X}\times \mathcal{U}\times \mathcal{Z}\rightarrow \mathbb{R}$ be a given function. 
Assume a joint distribution $p(x,z)$ is given.
Then the optimal solution $q^*(u|x, z)$ to a convex optimization problem
\[
\min_{q(u|x,z)} \mathbb{E}c_t(X,U,Z)+I(X;U|Z)
\]
is given by \cite{csiszar1974extremum}
\begin{align*}
q^*(u|x,z)&=\frac{\nu^*(u|z)\exp\{-c(x,u,z)\}}{\phi^*(x,z)} \\
\phi^*(x,z)&=\sum_{\mathcal{U}} \nu^*(u|z)\exp\{-c(x,u,z)\} \\
\nu^*(u|z)&=\sum_{\mathcal{X}} p(x|z)q^*(u|x,z).
\end{align*}
Moreover, the optimal value is $\mathbb{E}^{p(x,z)}\{-\log \phi^*(X,Z)\}$.

\subsection{Formulation as a dynamic programming problem}
We now solve the optimization problem \eqref{eqmainproblem} using dynamic programming. We assume the initial condition $\mu_0(x_0)=p_0(x_0)$ of the Markovian dynamics is given. In what follows, we assume $\beta=1$ without loss of generality. For each $t=0, 1, ... , T-1$, introduce the value function
\begin{align*}
&V_t(\mu_t(x^t, u^{t-1})):= \\
& \min_{\{q_k\}_{k=t}^{T-1}} \sum_{k=t}^{T-1} \mathbb{E}c_t(X_t, U_t, X_{t+1})+I(\bar{X}^t; U_t|U^{t-1}, \tilde{X}^t).
\end{align*}
The value function must satisfy the Bellman equation
\begin{align}
&V_t(\mu_t(x^t, u^{t-1}))= \nonumber \\
&\min_{q_t} \Bigl\{ \mathbb{E}c_t(X_t, U_t, X_{t+1})+ I(\bar{X}^t; U_t|U^{t-1}, \tilde{X}^t) \Bigr. \nonumber \\
&\hspace{20ex}\Bigl. + V_{t+1}(\mu_{t+1}(x^{t+1}, u^t)) \Bigr\} \label{eqbellman}
\end{align}
with the terminal condition
\[
V_T(\mu_T(x^T, u^{T-1}))=\mathbb{E}^{\mu_T} c_T(X_T).
\]
\begin{lemma}\label{lemvalue}
For each $t=0, 1, \cdots , T$, there exists a function $\phi_t(\cdot)$ such that
$V_t(\mu_t(x^t, u^{t-1}))=\mathbb{E}^{\mu_t}\{-\log \phi_t(X_t, U^{t-1})\}$.
\end{lemma}
\begin{proof}
Proof by induction. If $t=T$, the claim holds by choosing $\phi_T(x_T, u^{T-1})=\exp\{-c_T(x_T)\}$.
Thus, assume that there exists a function $\phi_{t+1}$ such that
\[
V_{t+1}(\mu_t(x^{t+1}, u^{t}))=\mathbb{E}^{\mu_{t+1}}\{-\log \phi_{t+1}(X_{t+1}, U^{t})\}.
\]
Then, the right hand side of the Bellman equation \eqref{eqbellman} becomes an optimization problem
\begin{align}
\min_{q_t}\quad &\mathbb{E}^{\mu_t, q_t, p_{t+1}}c_t(X_t, U_t, X_{t+1}) + I(\bar{X}^t; U_t|U^{t-1}, \tilde{X}^t) \nonumber \\
&+\mathbb{E}^{\mu_t, q_t, p_{t+1}}\{-\log \phi_{t+1}(X_{t+1}, U^t)\}. \label{eqopt1}
\end{align}
Introducing a function 
\begin{align*}
\rho_t(x_t, u^t)=&\sum_{\mathcal{X}_{t+1}} p_{t+1}(x_{t+1}|x_t, u_t) \\
&\times \{c_t(x_t, u_t, x_{t+1})-\log \phi_{t+1}(x_{t+1},u^t)\},
\end{align*}
\eqref{eqopt1} can be written as
\[
\min_{q_t} \quad \mathbb{E}^{\mu_t, q_t} \rho_t(\bar{X}_t, \tilde{X}_t, U^t)+I(\bar{X}^t; U_t|U^{t-1}, \tilde{X}^t).
\]
Considering $\bar{X}^t$ as $X$, $U_t$ as $U$, and $(U^{t-1}, \tilde{X}^t)$ as $Z$, the result in Section~\ref{secsinglestage} can be applied. Namely, the optimal solution is given by
\begin{align*}
q_t^*(u_t|x^t, u^{t-1})&=\frac{\nu_t^*(u_t|\tilde{x}^t, u^{t-1})\exp\{-\rho_t(x_t, u^t)\}}{\phi^*(x^t, u^{t-1})} \\
\phi^*(x^t, u^{t-1})&=\sum_{\mathcal{U}_t} \nu_t^*(u_t|\tilde{x}^t, u^{t-1})\exp\{-\rho_t(x_t, u^t)\} \\
\nu_t^*(u_t|\tilde{x}^t, u^{t-1})&=\sum_{\bar{\mathcal{X}}^t} \mu_t(\bar{x}^t|u^{t-1}, \tilde{x}^t)q^*(u_t|x^t, u^{t-1}).
\end{align*}
Moreover, the optimal value \eqref{eqopt1} can be written as
\begin{equation}
\label{eqoptvalue}
\mathbb{E}^{\mu_t} \{-\log\phi_t^*(x^t, u^{t-1})\}.
\end{equation}
Thus, we have constructed a function $\phi_t^*(x^t, u^{t-1})$ such that the right hand side of the Bellman equation  \eqref{eqbellman} can be written as \eqref{eqoptvalue}.
\end{proof}
\begin{theorem}
Suppose there exists a solution $(\mu^*, \nu^*, \rho^*, \phi^*, q^*)$ satisfying the following set of nonlinear equations
\begin{align*}
\mu_{t+1}^*(x^{t+1}, u^t)=&p_{t+1}(x_{t+1}|x_t, u_t)q_t^*(u_t|x^t, u^{t-1})\\
&\times \mu_t^*(x^t,u^{t-1}) \\
\nu_t^*(u_t|\tilde{x}^t, u^{t-1})=&\sum_{\bar{\mathcal{X}}^t}\mu_t(\bar{x}^t|u^{t-1}, \tilde{x}^t)q_t^*(u_t|x^t,u^{t-1}) \\
\rho_t^*(x_t, u^t)=&\sum_{\mathcal{X}_{t+1}}p_{t+1}(x_{t+1}|x_t, u_t) \\
&\times \{c_t(x_t, u_t, x_{t+1})-\log \phi_{t+1}(x_{t+1}, u^t)\} \\
\phi_t^*(x^t, u^{t-1})=&\sum_{\mathcal{U}_t} \nu_t^*(u_t|\tilde{x}^t, u^{t-1})\exp\{-\rho_t^*(x_t, u^t)\} \\
q_t^*(u_t|x^t, u^{t-1})=&\frac{\nu_t^*(u_t|\tilde{x}^t, u^{t-1})\exp\{-\rho_t^*(x_t, u^t)\}}{\phi_t^*(x^t, u^{t-1})}
\end{align*}
for each $t=0, 1, ..., T-1$ with the initial condition $\mu_0(x_0)=p_0(x_0)$ and the terminal condition $\phi_T(x_T, u^{T-1})=\exp\{-c_T(x_T)\}$. Then $\{q_t^*\}_{t=0}^{T-1}$ is the optimal solution to \eqref{eqmainproblem}.
\end{theorem}
\begin{proof}
	Suppose $(\mu^*, \nu^*, \rho^*, \phi^*, q^*)$ satisfy the above set of equations. Then, from the argument in the proof of Lemma~\ref{lemvalue}, the sequence of strategies $\{q_t^*\}_{t=0}^{T-1}$ solves Bellman equation \eqref{eqbellman} along the trajectory  $\{\mu_t^*\}_{t=0}^{T-1}$. Thus $\{q_t^*\}_{t=0}^{T-1}$ is an optimal solution to \eqref{eqmainproblem}.
\end{proof}

%%%%%%%%%%%%%%%%%%%%%%%%%%%%%%%%%%%%%%%%%%%%%%%%%%%%%%%%%%%%%%%%%%%%%%%%%%%%%%%%


\subsection{Arimoto-Blahut algorithm}
The optimality conditions derived in the previous section is a set of coupled non-linear equations with respect to the variables $\mu^{*},\nu^*,\rho^*,\phi^*,q^*$. In order to solve these we propose a numeric forward-backward algorithm. Firstly, note that if $\rho^*,\phi^*,q^*$ are known, $\mu^{*},\nu^*$ can be solved forwards in time. 

We refer the reader to \cite{takashi17} for more details on convergence results of the algorithm. 

%%%%%%%%%%%%%%%%%%%%%%%%%%%%%%%%%%%%%%%%%%%%%%%%%%%%%%%%%%%%%%%%%%%%%%%%%%%%%%%%

%\section{Interpretation of directed information}\label{sec:ab}
%In this section we present a physical intepretation of the use of transfer entropy in this problem formulation in the case of a networked control problem. We will prove the rate of communication $R(D)$ will be lower bounded by the directed information. We first prove the following lemma.
\begin{lemma}
\begin{align}
I(X_1^T \rightarrow U^T || X_2^T) \leq I(X_1^T \rightarrow W^T || U^{T-1},X_2^T)
\end{align}
\end{lemma}
\begin{proof}
\begin{align*}
0 &\leq I(X_1^T \rightarrow U^T || X_2^T) - I(X_1^T \rightarrow W^T || U^{T-1},X_2^T) \\
& = \sum_{t=1}^T I\left(X_1^t;W_t|W^{t-1},U^{t-1},X_2^t \right) - I\left(X_1^t;U_t|U^{t-1},X_2^t \right)\\
& = \sum_{t=1}^T I\left(X_1^t;W_t,U_t|W^{t-1},U^{t-1},X_2^t \right) \\&- I\left(X_1^t;U_t|U^{t-1},X_2^t \right)\\
& = \sum_{t=1}^T I\left(X_1^t;W_t|U^{t},X_2^t \right) - I\left(X_1^t;W^{t-1}|U^{t-1},X_2^t \right)\\
& = \sum_{t=1}^T I\left(X_1^t;W_t|U^{t},X_2^t \right) - I\left(X_1^{t-1};W^{t-1}|U^{t-1},X_2^{t-1} \right)\\
& = I\left(X_2^T;W^T | U^T,X_1^T \right) \geq 0 
\end{align*}
\end{proof}
The third line follows from the fact 
\begin{align*}
 I & \left(X_1^t;W_t,U_t| W^{t-1},U^{t-1}, X_2^t \right) =  \\ & I\left(X_1^t;W_t|W^{t-1},U^{t-1},X_2^t \right) + I\left(X_1^t;U_t|W^{t},U^{t-1},X_2^t \right) = \\ & I\left(X_1^t;W_t|W^{t-1},U^{t-1},X_2^t \right)
\end{align*}
The next line follows from using the chain rule. 

We now present one of the main results of the paper. 
\begin{theorem}
$R(D) \geq I(X_1^T \rightarrow U^T || X_2^T)$
\end{theorem}

\begin{proof}
\begin{align*}
\sum_{t=1}^T R_t &\geq \sum_{t=1}^T H(W_t)\\
&\geq \sum_{t=1}^T H(W_t\vert W^{t-1},U^{t-1},X_2^t)\\
&\geq \sum_{t=1}^T H(W_t\vert W^{t-1},U^{t-1},X_2^t) - H(W_t\vert X_1^t, W^{t-1},U^{t-1},X_2^t)\\
& =  \sum_{t=1}^T I \left(X_2^T \rightarrow W^T \vert \vert U^{T-1},X_2^{T} \right)\\
& \geq I \left(X_1^T \rightarrow U^T \vert \vert X_2^T \right)
\end{align*}
\end{proof}


\section{Experiment results}\label{sec:exp}
\subsection{Moving obstacle}
We first look at a small motion planning problem shown in figure \ref{fig:exp1} where the agent is tasked with reaching the goal state in green whilst avoiding collisions with the red static obstacles and a brown moving obstacle that moves in the area shown.
\begin{figure}
\centering
\includegraphics[scale = 0.5]{mdp_moveobs.png}
\caption{Gridworld with a moving obstacle}\label{fig:exp1}
\end{figure}
The state space of the agent is $(x,y,x_{obs},y_{obs})$ where the last two coordinates form the position of the moving obstacle. We assume that the state of the moving obstacle to be expensive to observe. Intuitively we expect to see if that we set the $\beta$ parameter high, \ie if the cost of information is high, the agent will go the long way around the wall as it will be too expensive to observe the moving obstacle. If the agent knows the position of the moving obstacle at all times, it can easily avoid collision as the obstacle motion is deterministic.

%%%%%%%%%%%%%%%%%%%%%%%%%%%%%%%%%%%%%%%%%%%%%%%%%%%%%%%%%%%%%%%%%%%%%%%%%%%%%%%%

\section{Conclusion and future work}
In this paper, we presented a formal way to integrate co-safe LTL constraints into a minimal-information MDP problem. This is the first step in analyzing temporal logic constraints in communication constrained problems. For future work, we aim to relax the co-safe requirement to allow more general classes of LTL formulas by analyzing the mean information cost over an infinite run. Furthermore, we aim to extend this work to a multiple coordinating agent formulation as this problem setting naturally lends itself to minimizing communication between agents who are trying to satisfy a joint specification. 



\bibliographystyle{IEEEtran}
\bibliography{main}

\appendices
\vspace{-0.15cm}
%\section{Proof of Theorem 5.1} \label{sec:prf}
%We first prove the following lemma.
\begin{lemma}
\label{lemcoding}
For each $T=1, 2, ...$, we have
\[
I(\bar{X}^{T} \rightarrow W^{T} || U^{T-1},\tilde{X}^{T}) \geq I(\bar{X}^{T} \rightarrow U^{T} || \tilde{X}^{T}).
\]
\end{lemma}
\begin{proof}
The result is established by the following chain of inequalities.
\begin{align*}
&  I(\bar{X}^T \rightarrow W^T || U^{T-1},\tilde{X}^T) - I(\bar{X}^T \rightarrow U^T || \tilde{X}^T)  \\
& = \sum_{t=0}^T I\left(\bar{X}^t;W_t|W^{t-1},U^{t-1},\tilde{X}^t \right) - I\left(\bar{X}^t;U_t|U^{t-1},\tilde{X}^t \right)\\
& \stackrel{\text{(a)}}{=} \sum_{t=0}^T I\left(\bar{X}^t;W_t,U_t|W^{t-1},U^{t-1},\tilde{X}^t \right) \\
& \hspace{10ex}- I\left(\bar{X}^t;U_t|U^{t-1},\tilde{X}^t \right)\\
& \stackrel{\text{(b)}}{=} \sum_{t=0}^T I\left(\bar{X}^t;W^t|U^{t},\tilde{X}^t \right) - I\left(\bar{X}^t;W^{t-1}|U^{t-1},\tilde{X}^t \right)\\
& \stackrel{\text{(c)}}{=} \sum_{t=0}^T I\left(\bar{X}^t;W^t|U^{t},\tilde{X}^t \right) - I\left(\bar{X}^{t-1};W^{t-1}|U^{t-1},\tilde{X}^t \right)\\
& \stackrel{\text{(d)}}{\geq} \sum_{t=0}^T I\left(\bar{X}^t;W^t|U^{t},\tilde{X}^t \right) - I\left(\bar{X}^{t-1};W^{t-1}|U^{t-1},\tilde{X}^{t-1} \right)\\
& \stackrel{\text{(e)}}{=} I\left(\bar{X}^T;W^T | U^T,\tilde{X}^T \right) \geq 0. 
\end{align*}
Equality (a) holds since
\begin{align*}
&I(\bar{X}^t; W_t, U_t|W^{t-1}, U^{t-1}, \tilde{X}^t)=\\
&I(\bar{X}^t; W_t|W^{t-1}, U^{t-1}, \tilde{X}^t)
+I(\bar{X}^t; U_t|W^t, U^{t-1}, \tilde{X}^t)
\end{align*}
and the second term is zero since $U_t$ is conditionally independent of $\bar{X}^t$ given $(W^t, U^{t-1})$. Equality (b) can be shown by applying the chain rule for mutual information in two different ways:
\begin{align*}
&I(\bar{X}^t; W^t, U_t|U^{t-1}, \tilde{X}^t) \\
&=I(\bar{X}^t; W^{t-1}|U^{t-1}, \tilde{X}^t)+I(\bar{X}^t; W_t, U_t| W^{t-1}, U^{t-1}, \tilde{X}^t) \\
&=I(\bar{X}^t; U_t|U^{t-1}, \tilde{X}^t)+I(\bar{X}^t;W^t|U^t, \tilde{X}^t).
\end{align*}
Equality (c) holds since
\begin{align*}
&I(\bar{X}^t; W^{t-1}|U^{t-1}, \tilde{X}^t)=\\
&I(\bar{X}^{t-1}; W^{t-1}|U^{t-1}, \tilde{X}^t)+I(\bar{X}_t; W^{t-1}|U^{t-1}, \tilde{X}^t, \bar{X}^{t-1})
\end{align*}
and the second term is zero. To see (d), apply the chain rule in two different ways:
\begin{align*}
&I(\bar{X}^{t-1}, \tilde{X}_t; W^{t-1}|U^{t-1},\tilde{X}^{t-1}) \\
&=I(\tilde{X}_t; W^{t-1}|U^{t-1}, \tilde{X}^{t-1})+I(\bar{X}^{t-1}; W^{t-1}|U^{t-1}, \tilde{X}^t) \\
&=I(\bar{X}^{t-1}; W^{t-1}|U^{t-1}, \tilde{X}^{t-1}) \\
&\hspace{5ex}+I(\tilde{X}_t;W^{t-1}|U^{t-1},\tilde{X}^{t-1}, \bar{X}^{t-1}).
\end{align*}
Since the last term is zero, we have 
\[
I(\bar{X}^{t-1}; W^{t-1}|U^{t-1}, \tilde{X}^{t-1}) \geq I(\bar{X}^{t-1}; W^{t-1}|U^{t-1}, \tilde{X}^t).
\]
Finally, (e) is due to the telescopic cancellation.
\end{proof}

Theorem~\ref{theorate} can now be shown as
\begin{align*}
R&=\sum_{t=0}^{T-1} R_t \\
&\stackrel{\text{(a)}}{\geq} \sum\nolimits_{t=0}^{T-1} H(W_t) \\
&\stackrel{\text{(b)}}{\geq} \sum\nolimits_{t=0}^{T-1} H(W_t|W^{t-1}, U^{t-1}, \tilde{X}^t) \\
&\stackrel{\text{(c)}}{\geq} \sum\nolimits_{t=0}^{T-1} H(W_t|W^{t-1}, U^{t-1}, \tilde{X}^t) \\
&\hspace{10ex}-H(W_t|\bar{X}^t, W^{t-1}, U^{t-1}, \tilde{X}^t) \\
&=\sum\nolimits_{t=0}^{T-1} I(\bar{X}^t; W_t|W^{t-1}, U^{t-1}, \tilde{X}^t) \\
&=I(\bar{X}^{T-1}\rightarrow W^{T-1}\| U^{T-2}, \tilde{X}^{T-1}) \\
&\stackrel{\text{(d)}}{\geq} I(\bar{X}^{T-1}\rightarrow U^{T-1}\| \tilde{X}^{T-1})
\end{align*}
Inequality (a) is a standard result - Theorem 2.6.4 in \cite{cover2012elements}. Inequality (b) holds since conditioning does not increase entropy, and (c) is because entropy is non-negative. Finally, (d) is due to Lemma~\ref{lemcoding}.



\end{document}
